\documentclass[hidelinks,11pt]{article}
\usepackage{hyperref}
\usepackage{xcolor}
\usepackage{tocloft}
\hypersetup {
 colorlinks,
 linkcolor={red!50!black},
 citecolor={red!50!black},
 urlcolor={red!80!black} 
}
\renewcommand{\refname}{}
\setlength{\parindent}{0pt}
\setlength{\parskip}{8pt}
%\setcounter{secnumdepth}{4}

\begin{document}
\title{LATTICE: Learning Attentional Traits In Tutoring Interactively for Cognitive Engagement}
\maketitle
\thispagestyle{empty}
\pagestyle{empty}

\pagebreak

%\renewcommand\thesection{Section \arabic{section}: }
%\renewcommand\thesubsection{Subsection \arabic{section}.\arabic{subsection}: }
%\renewcommand\thesubsubsection{Subsubsection \arabic{section}.\arabic{subsection}.\arabic{subsubsection}: }
%\renewcommand\theparagraph{Paragraph \arabic{section}.\arabic{subsection}.\arabic{subsubsection}.\arabic{paragraph}.}

\section{ Project Summary }

%\paragraph{}
In computer programming instruction, the nature of attention given during the
problem-solving process is a large factor in learning problem-solving skills
and in performance of students. However, cueing attention to the end of
facilitating problem-solving is a challenging task, as is formulating problems
which are engaging to students so as to direct attention to the problem-solving
process.  Intelligent tutoring systems (ITS) are computer-aided teaching and
assessment tools designed to aid and supplement course instruction; most ITS
systems provide adaptive tutoring, tailoring both the course content (facts,
examples, diagrams, etc.) and assessments (quizzes, tests, homeworks, etc.) to
students, in the hopes of engaging them.  However, there is to date no
comprehensive model of the mediating effect of attention on problem-solving in
an ITS. 

%\paragraph{}
The proposed project aims to construct such a model by utilizing established
EEG and eye-tracking measures of attention, and using known factors (cognitive
load, self-efficacy, domain-relevance) as a basis for further exploration. In
so doing, it also aims to incorporate the model into an ITS to provide a
practical tool for STEM instruction, which in addition to providing
auto-grading capability, is able to engage student interests and direct
atttention in a manner which maximally facilitates problem-solving.

\subsection{ Intellectual Merit }

%\paragraph{}
The proposed project aims to advance knowledge by identifying changes in
engagement due to variables associated with problems.  These include Bloom's
cognitive taxonomic level (knowledge, comprehension, application, analysis,
evaluation, synthesis), problem difficulty (easy, medium, hard), and domain
(biology, chemistry, physics, art, music, etc.).  We consider these per-concept
(for-loops, arrays, sorting algorithms, etc.).

%\paragraph{}
Furthermore, the project endeavors to explore the specific effects of problems
on engagement and to identify a means of scheduling problems and course
content, such that engagement is achieved in a way which maximally facilitates
problem-solving. Finally, it aims to leverage the unique potential of a
computer-aided tutoring system to assist in both conscious and unconscious
cognitive processes during problem-solving in a manner which traditional
instruction cannot. Specifically, masked priming has been shown to be effective
at raising self-efficacy \cite{jraidi2011}, and visual priming has been shown
to facilitate insight-based problem solving \cite{grant2003}.  The project will
aim to explore the limitations of this assistance.

\subsection{ Broader Impact }

%\paragraph{}
The project has the potential to assist in STEM instruction, particularly in
helping students overcome attention-related barriers and impasses during the
problem-solving process. The impact of the project extends but is not limited
to students of programming courses; the ITS framework aims to be adaptable to
other STEM (and even non-STEM) subjects.  A meta-analysis on computer-based
scaffolding in STEM has shown that it positively influences learning (with
Hedges g=.53) \cite{belland2015}.  If the ITS developed during the course of
this study should face barriers to adoption in the STEM instruction community,
the science gleaned from the proposed project could nonetheless aid in the
development of superior ITS frameworks.  The project also has the potential to
result in treatments to train attention in such a way that is conducive to
problem-solving, as has been done in studies which use biofeedback training
\cite{li2009}, \cite{li2011}, and programs to train spatial thinking for
STEM \cite{taylor2013}.

%\paragraph{}
The auto-grading component of LSU’s current ITS may be considered in its own
right to provide a broad positive impact by relieving teachers and their
assistants of the labor-intensive and time-consuming tasks involved in grading
(code or otherwise). As a necessary part of automated response set evaluation,
improving the accuracy of the auto-grading component would no doubt contribute
to STEM instruction: not only would it improve the efficiency of the grading
process, but it could be used to provide quicker feedback to students, who
benefit from the immediacy.  

%A meta-analysis on ITS concluded that ITS “outperformed, in aggregate, the
%other modes of instruction to which it was compared in evaluative studies”
%[citation]. It has been also shown that the inclusion of context (student
%background, previously solved problems etc.) in an ITS algorithm has a
%significant positive impact on student performance [citation].

%\paragraph{}

Self-efficacy is a major factor in engagement with material and problem-solving
performance, according to community perceptions \cite{vivian2014}. It has been
shown that self-efficacy in STEM disciplines differs between women and men
\cite{boy2013} \cite{gonzalez2012}, low-SES and high-SES groups
\cite{gonzalez2012}, and minority and majority groups; also that groups with
relatively low self-efficacy benefit from a mentoring program to raise it
\cite{macphee2013}.  In the case of women in particular, an examination of
long-term evidence suggests that initiatives have had little impact on the
gendered patterns of participation  \cite{smith2011}. Many women finish a
terminal bachelor's degree due to self-efficacy related reasons \cite{boy2013}.
It is possible that use of affective interventions with these groups at the
onset of an introductory-level course may improve participation and retention
\cite{jraidi2011}. 

%\paragraph{}
In addition, an intelligent tutoring system has the ability to aid students
with disabilities.  As one study by Hawley at al.  indicates, one of the
problems with transition from high school to higher education is ``restricted
access to facilities in STEM environments'': students with visual or speech
impairments often have barriers to participation in class which teachers do not
know how to accommodate; and some students with physical disabilities have lack
of access to reliable public transportation unless they live in the city
\cite{hawley2013} \cite{chapman2014}. An ITS can be used remotely by those
students with physical disabilities who could otherwise use computer
technology.


\pagebreak
\section{ References Cited }

\bibliography{main}
\bibliographystyle{apalike}
\end{document}

